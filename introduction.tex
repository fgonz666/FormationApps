% chapitre d'introduction
\chapter{Introduction}
%\addcontentsline{toc}{chapter}{Introduction}

Suite --ou en parallèle-- de la formation dispensée au collège, ce livret dont la vocation est de servir aussi bien de trame formative mais aussi de guide de référence au long terme puisqu'au fur et à mesure des révisions ce livret sera enrichi de nombreux chapitres consacrés aux applications indidividuelles présentes ou s'intégrant au bouquet de service offert sur apps.

Il est aussi possible de contribuer à l'élaboration du livre puisqu'il est présent sur un dépôt \emph{github} dont l'adresse figure à la fin de cet ouvrage. 

Où trouver de l'information ? C'est relativement simple, elle est disponible sur les salons du réseau social \textbf{tchap} dédiés à \emph{apps} : \url{https://matrix.to/#/#apps.education.fr:agent.education.tchap.gouv.fr} ouvert à toutes celles et ceux pouvant accéder au réseau.

Par choix, j'ai décidé d'articuler ce document de la manière suivante~:
\begin{itemize}
	\item cette introduction,
	\item la genèse et un court historique du projet,
	\item la connexion au service, que le compte soit déjà créé ou non,
	\item un survol sommaire des différents services logiciels proposé au sein du bouquet numérique,
	\item l'examen plus approfondi et précis des possibilités offertes par le nuage
	\item l'examen plus précis de chaque application au fur déjà présente
	\item viendront ensuite s'ajouter les examens plus précis des nouvelles applications ajoutées ...
	\item une table des contenus --chapitres, sections, fiches pratiques ...-- 
	\item les remerciements et informations légales
\end{itemize}

\paragraph{Participation.}
Le choix qui a été fait est de privilégier des langages d'écritures textuels les plus purs possibles, excluant \emph{de facto\/} les documents compressés bureautiques classiques (.doc, .docx, .odt). 
Cependant, dans un esprit de bienveillance et d'accueil des bonnes volontés, si vous savez enregistrer vos documents en ``document plat XML'' --disponibles avec \emph{Office} de Microsoft et \emph{libreoffice} de la Document Foundation-- ou bien, que dans un effort de formation vous vous initiez à l'utilisation de langages très simples tels que \emph{Markdown}\footnote{%
Descriptif du langage : \url{https://fr.wikipedia.org/wiki/Markdown} ; Un descriptif des quelques balises de formatage \url{https://www.ionos.fr/digitalguide/sites-internet/developpement-web/markdown/} et une petite leçon sur ce langage \url{https://programminghistorian.org/fr/lecons/debuter-avec-markdown}
}

Sinon, il est possible simplement de soumettre des modifications textuelles brutes en précisant le fichier à modifier, les lignes à rectifier et les modifications à apporter, ce qui est déjà une participation pertinente.

Vous pourrez retrouver une version en cours de développement de ce guide sur mon dépôt \emph{GitHub} à l'adresse \url{https://github.com/fgonz666/FormationApps}.

% Inclusion dans l'introduction : révisions du document
\section{Révisions du document}
\begin{table}
	\centering
	\renewcommand{\arraystretch}{1.25}
	\begin{tabular}{| m{0.15\linewidth} | m{0.1\linewidth} | m{0.15\linewidth} | m{0.55\linewidth} |}
		\hline
		Date & n$^0$ de version & Auteur principal & Modifications apportées \cr
		\hline
		2022-06-15 & 0.01 & F.S.G. & Création du document (Markdown) \cr
		\hline
		2022-06-17 & 0.02 & F.S.G. & Passage Markdown $\rightarrow$ \LaTeX{} \newline Plan modifié \cr
		\hline
		2022-06-18 & 0.03 & F.S.G. & Document monolithique $\rightarrow$ documents séparés pour une meilleure répartition des tâches et une lecture simplifiée. \cr
		\hline
		2022-06-20 & 0.04 & F.S.G. & Modification de la numérotation, ajout de nouvelles captures et de paragraphes sur l'utilisation du nuage. \cr
		\hline
		2022-06-24 & 0.05 & F.S.G. & Ajouts divers dans l'utilisation du nuage, tant en contenu et en structure.
		\newline Ajout d'une nouvelle capture d'écran sur CodiMD. \cr
		\hline
		2022-06-25 & 0.06 & F.S.G. & Modifications dans la partie libreoffice d'utiliser le nuage
		\newline Ajout d'une capture d'écran sur libreoffice writer. \cr
		\hline
	\end{tabular}
\end{table}

\paragraph{Objectif} : Initier les collègues à l'utilisation du portail apps mis en place par notre ministère pour tous les agents qu'ils soient en académie ou en administration centrale. 

\vspace{2cm}

Cette formation et son livret d'accompagnement s'organisent suivant le plan suivant~:
\begin{itemize}
    \item Création du compte
    \item Présentation des services
    \item Présentation plus détaillée du \emph{cloud\/}.
\end{itemize}
