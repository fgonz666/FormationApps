% chapitre historique
\chapter{Un peu d'histoire}
%\addcontentsline{toc}{chapter}{Un peu d'histoire}

\emph{Apps} est un projet initié en 2017 par les équipes du ministère (Bureau B1-1 de la direction du Numérique Éducatif) pour proposer aux agents un environnement numérique  moderne et répondant aux usages des utilisateurs, dans le cadre du projet SNP (Service Numériques Partagés). 
En effet la fragmentation des usages liée au nombre important des membres de l'éducation nationale --estimés dans le rapport à 1,2 millions d'agents dont environ 900~000 enseignants et 300~000 non-enseignants--, la diversité des métiers --une centaine répertoriés-- et par conséquent des usages rend difficile de proposer un outil unique avec un usage monolithique. 
Le choix d'un agréggateur de services permettant à chacun d'activer les services correspondant à ses usage a été alors préféré.

Ce projet a été ensuite récupéré par le bureau \og~Socle Numérique~\fg{} afin d'offrir aux agents des outils numériques de travail collaboratif mais toujours avec l'optique d'outils numérique, collaboratifs et accessibles sur le lieu de travail ou à l'extérieur, domicile compris. ``Un commun pour les agents''.

Le service centralisateur, appelé ``La Boite'',  est situé ici : \url{https://portail.apps.education.fr}

