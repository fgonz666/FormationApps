% chapitre dédié au nuage
\chapter*{Utiliser le nuage}
\addcontentsline{toc}{chapter}{Utiliser le nuage}

Le ``Nuage'' est le nom qui a été donné à l'espace de stockage en ligne, actuellement d'une taille maximale de 100~Go\footnote{%
100~Go = 100 milliards de caractères simples.
} 
offrant en outre la présence d'une suite bureautique en ligne --libreoffice en ligne ou lool-- et d'éditeurs avancés de texte enrichi.

Le nuage est accessible de deux façons différentes pour l'instant --j'espère que la troisième fonctionnera aussi bientôt--~: \emph{via} l'interface web (firefox, chrome, edge, autres ...) et \emph{via} le client \emph{nextcloud}.

\section*{L'interface Web}
\addcontentsline{toc}{section}{L'interface web}

L'interface web est celle qui sera sans doute la plus utilisée car dès que l'on est en établissement scolaire, peu importe les blocages ou les applications, la certitude de trouver un navigateur dans chaque ordinateur est quasi absolue.
\begin{figure}
	\centering
	\includegraphics[width=\linewidth]{./Captures/nuage.accueil.png}
	\caption{Le nuage en mode détaillé}
\end{figure}
L'interface est visible en mode mosaïque également.
\begin{figure}
	\centering
	\includegraphics[width=\linewidth]{./Captures/nuage.accueil.mozaique.png}
	\caption{Le nuage en mode mosaïque.}
\end{figure}
L'intérêt de cette interface est de pouvoir y déposer ou récupérer un ou plusieurs fichiers, ou bien un ou plusieurs dossiers. 
L'autre intérêt est que le site étant en ``education.fr'' il ne sera donc pas filtré par les systèmes de pare-feu académiques et évitera l'emploi de clés USB qui se promènent entre le domicile et l'établissement où les niveau de sécurité sont très différents.

\section*{Le client NextCloud}
\addcontentsline{toc}{section}{Le client nextcloud}

Le nuage de \emph{apps} est basé sur le travail d'un groupe important de développeurs et de développeuses qui ont fondé le site et le serveur ``Nextcloud''. 
Ce service en ligne permet outre ce qui est offert par le nuage bien d'autres fonctionnalités. 

Outre la connexion via une interface web, un `` client'' pour ordinateurs et pour \emph{smartphones} a lui aussi été développé.

\section*{Transfert de données}
\addcontentsline{toc}{section}{Transfert de données}

bla lbla

\subsection*{Monter un ou plusieurs fichiers}
\addcontentsline{toc}{subsection}{Monter un ou plusieurs fichiers}

bla lbla

\subsection*{Créer un dossier}
\addcontentsline{toc}{subsection}{Créer un dossier}

bla lbla

\subsection*{Descendre un dossier ou plusieurs fichiers}
\addcontentsline{toc}{subsection}{Descendre un dossier ou plusieurs fichiers}

bla lbla

\subsection*{Ce qu'il n'est pas possible de faire}
\addcontentsline{toc}{subsection}{Ce qu'il n'est pas possible de faire} 

bla lbla

\section*{Partager une ressource}
\addcontentsline{toc}{section}{Partager une ressource} 

Ressource = élément unique

\subsection*{Partage simple}
\addcontentsline{toc}{subsection}{Partage simple}

\begin{itemize}
    \item Vers un utilisateur
    \item Vers un groupe
    \item Vers le public 
\end{itemize}

\subsection*{Les options de partage avancées}
\addcontentsline{toc}{subsection}{Les options du partage}

bla lbla