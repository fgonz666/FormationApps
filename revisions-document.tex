% Inclusion dans l'introduction : révisions du document
\section{Révisions du document}
\begin{table}
	\centering
	\renewcommand{\arraystretch}{1.25}
	\begin{tabular}{| m{0.15\linewidth} | m{0.1\linewidth} | m{0.15\linewidth} | m{0.55\linewidth} |}
		\hline
		Date & n$^0$ de version & Auteur principal & Modifications apportées \cr
		\hline
		2022-06-15 & 0.0.1 & F.S.G. & Création du document (Markdown) \cr
		\hline
		2022-06-17 & 0.0.2 & F.S.G. & Passage Markdown $\rightarrow$ \LaTeX{} \newline Plan modifié \cr
		\hline
		2022-06-18 & 0.0.3 & F.S.G. & Document monolithique $\rightarrow$ documents séparés pour une meilleure répartition des tâches et une lecture simplifiée. \cr
		\hline
		2022-06-20 & 0.0.4 & F.S.G. & Modification de la numérotation, ajout de nouvelles captures et de paragraphes sur l'utilisation du nuage. \cr
		\hline
		2022-06-24 & 0.0.5 & F.S.G. & Ajouts divers dans l'utilisation du nuage, tant en contenu et en structure.
		\newline Ajout d'une nouvelle capture d'écran sur CodiMD. \cr
		\hline
		2022-06-25 & 0.0.6 & F.S.G. & Modifications dans la partie libreoffice d'utiliser le nuage
		\newline Ajout d'une capture d'écran sur libreoffice writer. \cr
		\hline
		2022-06-26 & 0.0.7 & F.S.G. & Gros ajout sur les profils et les groupes et les applications et les bookmarks ainsi que pour l'ajout de media et aussi sur le chapitre de l'agenda. \cr
		\hline
	\end{tabular}
\end{table}